\chapter{Brain Teasers}

\begin{exe}[Chess tournament]
A chess tournament has $2^n$ players with skills $1>2>\cdots>2^n$. It is organized as a
knockout tournament, so that after each round only the winner proceeds to the next
round. Except for the final, opponents in each round are drawn at random. Let's also
assume that when two players meet in a game, the player with better skills always wins.
What's the probability that players 1 and 2 will meet in the final?
\end{exe}
\begin{teacher}
Source: Practical Guide
\begin{sol}
\end{sol}
\end{teacher}

\begin{exe}[100th digit]
What is the 100-th digit of $(1+\sqrt{2})^{3000}$? Equivalently, what is the unit digit of $(1+\sqrt{2})^{3000}*10^{100}$?\\
Hint: $(1+\sqrt{2})^2 + (1-\sqrt{2})^2=6$
\end{exe}
\begin{teacher}
Source: Practical Guide
\begin{sol}
With the hint, we claim that $(1+\sqrt{2})^{3000} + (1-\sqrt{2})^{3000}$ is an integer. (Proofs needed?)\\
Also we have $|1-\sqrt{2}|<\frac{1}{2}$, hence $(1-\sqrt{2})^{3000}< \frac{1}{2^{3000}}<<10^{-100}$. So the 100th digit for $(1-\sqrt{2})^{3000}$ is 0, hence the 100th digit of $(1+\sqrt{2})^{3000}$ is 9.
\end{sol}
\end{teacher}

\begin{exe}
How many numbers $x$ are there less that $10^6$ that the cubic of $x$ ends with 11, i.e. $x^3\ mod\  100=11$?
\end{exe}
\begin{teacher}
Source: Practical Guide
\begin{sol}
\end{sol}
\end{teacher}

\begin{exe}
Given two bottles of liquid of volume $V$. One is water and other is pure alcohol. You have a spoon of volume $U$, pour water into alcohol then the mixed liquid back to the water. You can assume the bottles are large enough so that no liquid will leak, both time you transfer $U$ volume full without left. The questions here is, which is higher, the percentage of water in alcohol bottle, or the percentage of alcohol in water bottle. 
\end{exe}
\begin{teacher}
\begin{sol}
\end{sol}
\end{teacher}

\begin{exe}
There are two bells. One rings 4 times per minute while the other rings 5 time per minute. You here them ring at the same time. How long you need to wait until they ring at the same time again?
\end{exe}
\begin{teacher}
\begin{sol}
\end{sol}
\end{teacher}

\begin{exe}
What is the sum of integers from 1 to 100?
\end{exe}
\begin{teacher}
\begin{sol}
\end{sol}
\end{teacher}

\begin{exe}
A clock face is break into three parts. The numbers in each part has the same sum. How the faces is broken?
\end{exe}
\begin{teacher}
\begin{sol}
\end{sol}
\end{teacher}

\begin{exe}
Given 12 marbles and a weight. The weight have two sides and only can tell which side is heavier or both side equal. 1 marble is fake and has different weight (may be heavier or lighter). How can you tell which marble is fake by using the weight only 3 times?
\end{exe}
\begin{teacher}
\begin{sol}
\end{sol}
\end{teacher}

\begin{exe}
On Cartesian plane, a circle is tangent to the x-axis and y-axis. A point on the circle has coordinate $(5, 10)$. What is the radius of the circle?
\end{exe}
\begin{teacher}
\begin{sol}
\end{sol}
\end{teacher}

\begin{exe}
On a cubic room, a bug is sitting in one corner. What is the shortest way to move to the extreme opposite corner?
\end{exe}
\begin{teacher}
\begin{sol}
\end{sol}
\end{teacher}

\begin{exe}
On a 10 by 10 by 10 cube, the outer surface is painted. How many cubes are painted at least 1 face?
\end{exe}
\begin{teacher}
\begin{sol}
\end{sol}
\end{teacher}

\begin{exe}
You have 10 tanks, each equipped with 1000L gasoline. A tank will consume 1L gasoline per mile. Each tank cannot carry more than 1000L, but you are allowed to transfer gasoline from 1 tank to the other in the middle way. For example, after going 100 miles, you can transfer 100L from tank A to tank B, so tank A has 800 and tank B has 1000. What is the strategy to reach the most distance from start and how many miles can you reach?
\end{exe}
\begin{teacher}
\begin{sol}
\end{sol}
\end{teacher}

\begin{exe}
What is the angle between the hour and minute hands of a clock at the time of 6.15?
\end{exe}
\begin{teacher}
\begin{sol}
\end{sol}
\end{teacher}

\begin{exe}
What time is it that after 3:00, the hour hand and minute hand coincide the first time?
\end{exe}
\begin{teacher}
\begin{sol}
\end{sol}
\end{teacher}

\begin{exe}
In a room, there are 100 bulbs and 100 players, both labeled a different numbers from 1 to 100. Player i will change the state of the bulbs with numbers which is a multiple of i. Changing the state means switch on if the bulb is off and switch off if the bulb is on. For example, player 10 will change the state of 10, 20, ... 100. After all the players have done their operations, which bulbs are on?
\end{exe}
\begin{teacher}
\begin{sol}
\end{sol}
\end{teacher}

\begin{exe}

\end{exe}
\begin{teacher}
\begin{sol}
\end{sol}
\end{teacher}

\begin{exe}

\end{exe}
\begin{teacher}
\begin{sol}
\end{sol}
\end{teacher}

\begin{exe}

\end{exe}
\begin{teacher}
\begin{sol}
\end{sol}
\end{teacher}

\begin{exe}

\end{exe}
\begin{teacher}
\begin{sol}
\end{sol}
\end{teacher}

\begin{exe}

\end{exe}
\begin{teacher}
\begin{sol}
\end{sol}
\end{teacher}

\begin{exe}

\end{exe}
\begin{teacher}
\begin{sol}
\end{sol}
\end{teacher}

\begin{exe}

\end{exe}
\begin{teacher}
\begin{sol}
\end{sol}
\end{teacher}

\begin{exe}

\end{exe}
\begin{teacher}
\begin{sol}
\end{sol}
\end{teacher}

\begin{exe}

\end{exe}
\begin{teacher}
\begin{sol}
\end{sol}
\end{teacher}

\begin{exe}

\end{exe}
\begin{teacher}
\begin{sol}
\end{sol}
\end{teacher}

\begin{exe}

\end{exe}
\begin{teacher}
\begin{sol}
\end{sol}
\end{teacher}

\begin{exe}

\end{exe}
\begin{teacher}
\begin{sol}
\end{sol}
\end{teacher}

\begin{exe}

\end{exe}
\begin{teacher}
\begin{sol}
\end{sol}
\end{teacher}

\begin{exe}

\end{exe}
\begin{teacher}
\begin{sol}
\end{sol}
\end{teacher}

\begin{exe}

\end{exe}
\begin{teacher}
\begin{sol}
\end{sol}
\end{teacher}

\begin{exe}

\end{exe}
\begin{teacher}
\begin{sol}
\end{sol}
\end{teacher}

\begin{exe}

\end{exe}
\begin{teacher}
\begin{sol}
\end{sol}
\end{teacher}

\begin{exe}

\end{exe}
\begin{teacher}
\begin{sol}
\end{sol}
\end{teacher}

\begin{exe}

\end{exe}
\begin{teacher}
\begin{sol}
\end{sol}
\end{teacher}

\begin{exe}

\end{exe}
\begin{teacher}
\begin{sol}
\end{sol}
\end{teacher}













% \begin{exe}

% \end{exe}
% \begin{teacher}
% \begin{sol}
% \end{sol}
% \end{teacher}
