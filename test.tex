\documentclass[a4paper,10pt]{article}
\usepackage{a4wide}
\usepackage[english]{babel}
\usepackage[copyexercisesinsolutions]{exsol}
\title{Gobal example, from the \textsf{ExSol} package}
\author{Walter Daems}
\setlength{\parindent}{0em}

\begin{document}
\maketitle
\section{Introduction}
In this text we explain how to solve second-order polynomial
equations.
\section{Solving second-order polynomial equations}
\begin{informulacollectiononly}
\section*{Solving second-order polynomial equations}
\end{informulacollectiononly}
\begin{informulacollection}
The roots of the following equation
\begin{equation}
a x^2 + bx + c = 0
\end{equation}
can be determined as:
\begin{equation}
x_{1,2} = \frac{-b \pm \sqrt{b^2 - 4 a c}}{2 a}
\end{equation}
\end{informulacollection}
\begin{exercises}[columns = 2]
\begin{exercise}
Solve the following equation for $x \in C$, with $C$ the set of
complex numbers:
\begin{equation}
5 x^2 -3 x = 5
\end{equation}
\end{exercise}
\begin{solution}
Let’s start by rearranging the equation, a bit:
\begin{eqnarray}
5.7 x^2 - 3.1 x &=& 5.3\\
5.7 x^2 - 3.1 x -5.3 &=& 0
\end{eqnarray}
The equation is now in the standard form:
\begin{equation}
a x^2 + b x + c = 0
\end{equation}
For quadratic equations in the standard form, we know that two
solutions exist:
\begin{equation}
x_{1,2} = \frac{ -b \pm \sqrt{d}}{2a}
\end{equation}
with
\begin{equation}
d = b^2 - 4 a c
\end{equation}
5
If we apply this to our case, we obtain:
\begin{equation}
d = (-3.1)^2 - 4 \cdot 5.7 \cdot (-5.3) = 130.45
\end{equation}
and
\begin{eqnarray}
x_1 &=& \frac{3.1 + \sqrt{130.45}}{11.4} = 1.27\\
x_2 &=& \frac{3.1 - \sqrt{130.45}}{11.4} = -0.73
\end{eqnarray}
The proposed values $x = x_1, x_2$ are solutions to the given equation.
\end{solution}
\begin{exercise}
Consider a 2-dimensional vector space equipped with a Euclidean
distance function. Given a right-angled triangle, with the sides
$A$ and $B$ adjacent to the right angle having lengths, $3$ and
$4$, calculate the length of the hypotenuse, labeled $C$.
\end{exercise}
\begin{solution}
This calls for application of Pythagoras’ theorem, which
tells us:
\begin{equation}
\left\|A\right\|^2 + \left\|B\right\|^2 = \left\|C\right\|^2
\end{equation}
and therefore:
\begin{eqnarray}
\left\|C\right\|
&=& \sqrt{\left\|A\right\|^2 + \left\|B\right\|^2}\\
&=& \sqrt{3^2 + 4^2}\\
&=& \sqrt{25} = 5
\end{eqnarray}
Therefore, the length of the hypotenuse equals $5$.
\end{solution}
\end{exercises}
And now, we can come to conclusion.
\section{Conclusion}
Solving second-order polynomial equations is very easy.
\end{document}
